\documentclass{turgon}

%\usepackage{lmodern}

\usepackage[printwatermark]{xwatermark}
\newwatermark[allpages,color=black!15,angle=55,scale=5,xpos=0,ypos=0]%
{DRAFT}

%\doublespacing
\linespread{1.2}

\title{
%
Code Development for the Space-Time Conservation Element and Solution Element
Method
%
}

\author{
%
Yung-Yu Chen
%
}

%\date{2008.6.4}

\begin{document}

\maketitle

\tableofcontents

%%%%%%%%%%%%%%%%%%%%%%%%%%%%%%%%%%%%%%%%%%%%%%%%%%%%%%%%%%%%%%%%%%%%%%%%%%
%%
\chapter*{Introduction}
\addcontentsline{toc}{chapter}{Introduction}
%%
%%%%%%%%%%%%%%%%%%%%%%%%%%%%%%%%%%%%%%%%%%%%%%%%%%%%%%%%%%%%%%%%%%%%%%%%%%

Numerical computation, facilitated by the advancement of digital computers,
enables all kinds of simulation.  We use it to solve non-linear hyperbolic
partial differential equations (PDEs), which come from conservation laws
\citep{lax_hyperbolic_1973}.  This note describes a set of libraries that uses
the space-time conservation element and solution element (CESE) method to solve
the equations.  The libraries are written in modern C++ and provide Python
interface for interactive computing and prototyping.

% TODO: Basics of first-order hyperbolic partial differential equations:
% * d'Alembert solution
% * Method of characteristics
% * Non-linear equations
%   * Burger's equations
% * Riemann-invariant
%   * The shallow-water equations
%   * The Euler equations
% * Weak solutions

%%%%%%%%%%%%%%%%%%%%%%%%%%%%%%%%%%%%%%%%%%%%%%%%%%%%%%%%%%%%%%%%%%%%%%%%%%
%%
\chapter{The Conservation Element and Solution Element Method}
\label{c:cese}
%%
%%%%%%%%%%%%%%%%%%%%%%%%%%%%%%%%%%%%%%%%%%%%%%%%%%%%%%%%%%%%%%%%%%%%%%%%%%

The conservation element and solution element (CESE) method solves conservation
laws, which can be written in the following form in one-dimensional space
\begin{align}
  \frac{\partial u}{\partial t} + \frac{\partial f(u)}{\partial x} = 0
  .
  \label{e:cese:1d_pde}
\end{align}
$u$ is the dependent solution variable and $f(u)$ is a function.  $(x, t)$ is
the independent variables defining the two axes of two-dimensional Euclidean
space.  Let $\mathbf{h} \defeq (f(u),u)$ and rewrite Eq.~(\ref{e:cese:1d_pde})
to $\nabla\cdot\mathbf{h} = 0$ with the divergence operator.  Assuming
Eq.~(\ref{e:cese:1d_pde}) applies everywhere in the control volume $V$, we
write
\begin{align*}
  \int_V\nabla\cdot\mathbf{h}\dif v = 0
  .
\end{align*}
By using the divergence theorem, the above differential equation is cast into
an integral equation over the control surface $S(V)$ surrounding $V$
\begin{align}
  \oint_{S(V)}\mathbf{h}\cdot\dif\hat{\sigma} = 0
  .  \label{e:cese:1d_integral_form}
\end{align}
$\dif\hat{\sigma}$ is the infinitesimal surface vector.  Equation
(\ref{e:cese:1d_integral_form}) does not require the point-wise divergence free
condition, and it is what the CESE method solves \citep{chang_method_1995}.

It is important that the CESE method solves the integral equation, rather than
the differential equation, by enforcing the space-time flux conservation as
shown in Eq.~(\ref{e:cese:1d_integral_form}).  The solution variable and its
partial derivative in space are independent but solved together.  It uses a
compact stencil that defines two entities: the conservation elements (CEs) and
solution elements (SEs).  Space-time invariants are used to minimize numerical
dissipation, but the characteristics-based methods are not used for obtaining
solution.  Ad hoc treatments are avoided as much as possible.  The detail of
the theory can be found in \cite{chang_new_1991} and \cite{chang_method_1995}.

\begin{figure}[htbp]
\centering
  \includegraphics{cce.eps}
  \caption{A compounded conservation element (CCE), the area enclosed by the
  {\color{red} red} dots of {\color{red} $\square\mathrm{BCEF}$}, contains two
  basic conservation elements (BCEs), the area enclosed by the {\color{blue}
  blue} dots of {\color{blue} $\square\mathrm{ABCD}$} and {\color{blue}
  $\square\mathrm{ADEF}$}.}
  \label{f:cce}
\end{figure}

The CEs discretize the space-time for the integral equation to be solved
(Eq.~\ref{e:cese:1d_integral_form})).  $\mathrm{CE}(j,n)$ denotes a single CE
associated with the grid point $(x_j, t^n)$.  In the CE, the conservation of
$\mathbf{h}$ is approximated as
\begin{align}
  \oint_{S(\mathrm{CE})}\mathbf{h}^*\cdot d\hat{\sigma} = 0
  \label{e:conserv_of_approx_h}
\end{align}
where $\mathbf{h}^*$, which will be defined later with SEs, denotes the
approximation of $\mathbf{h}$.  As shown in \figurename~\ref{f:cce}, a CE
defined like that is a compounded conservation element (CCE), consisting of two
adjacent basic conservation elements (BCEs) $\mathrm{CE}_-$ and
$\mathrm{CE}_+$.  Equation (\ref{e:conserv_of_approx_h}) holds in both CCEs and
BCEs, i.e.,
\begin{align*}
  \oint_{S(\mathrm{CE}_\pm)}\mathbf{h}^*\cdot d\hat{\sigma} = 0 .
\end{align*}
$S(\mathrm{CE}_{\pm})$ is the bounding surface surrounding $\mathrm{CE}_{\pm}$.

\begin{figure}[hbtp]
  \centering
  \includegraphics{cese_marching.eps}
  \caption{Time-marching the solution by using the cross-shaped solution
  elements.  The {\color{red} red} dotted crosses mark the SEs at $t=t^0$.  The
  {\color{blue} blue} dotted crosses mark the SEs at $t=t^{1/2}$.  The
  {\color{orange} orange} dotted crosses mark the SEs at $t=t^1$.  The bigger
  dots at the horizontal middle points of the crosses are the solution points.
  The arrows show how the solution variable and its spatial derivative, that
  are defined at the solution points, at the previous half time step propagate
  to those at the next half time step.  The solution points at the boundary,
  $x=x_0$ and $x=x_4$, need to be updated by boundary-condition treatments,
  rather than the CESE method scheme.}
  \label{f:cese_marching}
\end{figure}

The SEs determine $\mathbf{h}^*$.  There is more than one way to define SEs,
while an effective and consistent approach is shown in
\figurename~\ref{f:cese_marching}.  Let $\mathrm{SE}(j,n)$ denote the SE
associated with the grid point $(j,n)$, which is the cross-shaped mark enclosed
by the dotted line.  The solution variable approximation is written as
\begin{align*}
  u^*(x,t;j,n) = u_j^n + (u_x)_j^n(x-x_j) + (u_t)_j^n(t-t^n) .
\end{align*}
The grid point $(x_j, t^n)$ is used as the solution point.  $u_j^n$,
$(u_x)_j^n$, and $(u_t)_j^n$ hold constant in $\mathrm{SE}(j,n)$.  It should be
noted that every CE is surrounded by SEs.  Fluxes evaluated through the CE
boundary depends only on the approximation within SEs.  To proceed, write
\begin{align*}
  \frac{\partial u^*(x,t;j,n)}{\partial x} = (u_x)_j^n, \quad
  \frac{\partial u^*(x,t;j,n)}{\partial t} = (u_t)_j^n .
\end{align*}
Substitute the approximated solution variable $u^*$ back to the original
differential equation (Eq.~(\ref{e:cese:1d_pde})) and obtain the relation
between $(u_x)_j^n$ and $(u_t)_j^n$ as
\begin{align*}
                  (u_t)_j^n + (f_u)_j^n(u_x)_j^n = 0
  \;\Rightarrow\; (u_t)_j^n = -(f_u)_j^n(u_x)_j^n .
\end{align*}
The approximated solution variable $u^*$ is then rewritten as
\begin{align*}
  u^*(x,t;j,n) = u_j^n + (u_x)_j^n\left[(x-x_j) - (f_u)_j^n(t-t^n)\right] .
\end{align*}
Similarly, the approximated function
\begin{align*}
  f^*(x,t;j,n) = f_j^n + (f_x)_j^n(x-x_j) + (f_t)_j^n(t-t^n)
\end{align*}
is rewritten as
\begin{align*}
  f^*(x,t;j,n) = f_j^n + (f_u)_j^n (u_x)_j^n \left[
    (x-x_j) - (f_u)_j^n(t-t^n)
  \right] .
\end{align*}

To this point, we are ready to write the time-marching formulae for the
$c$-scheme.  It is the most simple time-marching scheme for the CESE method.
The formulae include updating the solution variable and the spatial derivative.
The part for the solution variable $u$ will be obtained by enforcing the
space-time flux conservation over the CCE as shown in Fig.~\ref{f:se_flux}.  It
should be noted that the height of the CEs and SEs is the half time step
$\Delta t/2$, not the full time step $\Delta t$.

\begin{figure}[hbtp]
  \centering
  \includegraphics{se_flux.eps}
  \caption{Space-time flux at the boundary of $\mathrm{CE}(j, n+\frac{1}{2})$
  defined by $\mathrm{SE}(j-\frac{1}{2}, n)$ ({\color{red}red}),
  $\mathrm{SE}(j+\frac{1}{2}, n)$ ({\color{blue} blue}), and $\mathrm{SE}(j,
  n+\frac{1}{2})$ ({\color{orange} orange}).  $x_j$ denotes the grid point of
  the $j$-th SE, $x_j^{\pm}$ the right and left end points, $x_j^s$ the
  solution point.  $\Delta x_j^+ \defeq x_j^+ - x_j$ and $\Delta x_j^- \defeq
  x_j - x_j^-$ are the length of the right and left arms of the $j$-th SE.
  $(\mathbf{h^*})_{j,\pm}^n$ and $(\mathbf{h^*})_{j}^{n,+}$ are the arithmetic
  average of the right, left, and upper arm of $\mathrm{SE}(j,n)$,
  respectively.}
  \label{f:se_flux}
\end{figure}

\begin{figure}[hbtp]
  \centering
  \includegraphics{nonuni_se.eps}
  \caption{The SE definition for a non-uniform one-dimensional grid.  The
  cross-shaped marks are the SEs, and the solid dots are the associated
  solution points.}
  \label{f:nonuni_se}
\end{figure}

The solution point must be at the center of the SE to make first-order
approximation consistent.  In a non-uniform grid, the solution points may not
collocate with grid points (see Fig.~\ref{f:nonuni_se}).  The approximation
formulae should be changed to
\begin{align*}
  u^*(x,t;j,n) &= u_j^n + (u_x)_j^n \left[
    (x-x_j^s) - (f_u)_j^n(t-t^n)
  \right] , \\
  f^*(x,t;j,n) &= f_j^n + (f_u)_j^n (u_x)_j^n \left[
    (x-x_j^s) - (f_u)_j^n(t-t^n)
  \right] .
\end{align*}
The formula for the solution variable $u$ is found to be
\begin{align}
  u_j^{n+\frac{1}{2}}
    = \frac{1}{\Delta x_j}\left\{
      (u^*)_{j-\frac{1}{2},+}^n \Delta x_{j-\frac{1}{2}}^+
    + (u^*)_{j+\frac{1}{2},-}^n \Delta x_{j+\frac{1}{2}}^-
    + \frac{\Delta t}{2} \left[
        (f^*)_{j-\frac{1}{2}}^{n,+}
      - (f^*)_{j+\frac{1}{2}}^{n,+}
      \right]
      \right\}
  \label{e:formula:u}
\end{align}
where
\begin{align*}
  (u^*)_{j\mp\frac{1}{2},\pm}^n
   &= u_{j\mp\frac{1}{2}}^n
    + (u_x)_{j\mp\frac{1}{2}}^n
      \left( x_{j\mp\frac{1}{2}}
           \pm \frac{1}{2} \Delta x_{j\mp\frac{1}{2}}^{\pm}
           - x_{j\mp\frac{1}{2}}^s \right), \\
  (f^*)_{j\mp\frac{1}{2}}^{n,\pm}
   &= f_{j\mp\frac{1}{2}}^n
    + (f_u)_{j\mp\frac{1}{2}}^n(u_x)_{j\mp\frac{1}{2}}^n
      \left[x_{j\mp\frac{1}{2}} - x_{j\mp\frac{1}{2}}^s
          - (f_u)_{j\mp\frac{1}{2}}^n \frac{\Delta t}{4}
      \right] .
\end{align*}
See Fig.~\ref{f:nonuni_se} for the definition of the left-hand side of the
above equations.

The first-order derivative $u_x$ needs another formula.  The one of the
$c$-scheme is
\begin{align}
  (u_x)_j^{n+\frac{1}{2}} = \frac{
    (u')_{j+\frac{1}{2}}^n - (u')_{j-\frac{1}{2}}^n
  }{\Delta x_j}
  \label{e:formula:ux:c}
\end{align}
where
\begin{align*}
  (u')_{j\pm\frac{1}{2}}^n &\defeq
      u_{j\pm\frac{1}{2}}^n
    + (u_x)_{j\pm\frac{1}{2}}^n \left[
        x_{j\pm\frac{1}{2}} - x_{j\pm\frac{1}{2}}^s
      - (f_u)_{j\pm\frac{1}{2}}^n \frac{\Delta t}{2} \right]
\end{align*}
are the Taylor expansion to $t^{n+\frac{1}{2}}$ with respect to
$\mathrm{SE}(j\pm\frac{1}{2}, n)$.  Equations (\ref{e:formula:u}) and
(\ref{e:formula:ux:c}) together form the $c$-scheme.

A weighting function should be introduced to treat discontinuity in space.
Equations (\ref{e:formula:u}) and (\ref{e:formula:ux:c}) are called the
$c$-scheme because Eq.~(\ref{e:formula:ux:c}) uses central-differencing to
approximate $(u_x)_j^{n+\frac{1}{2}}$.  It is second-order accurate but doesn't
give correct result when the solution variable $u$ is discontinuous between
$x_{j-\frac{1}{2}}$ and $x_{j+\frac{1}{2}}$.  To define the weighting function,
let
\begin{align*}
  (u_{x\pm})_j^{n+\frac{1}{2}} \defeq
    \pm\frac{
      (u')_{j\pm\frac{1}{2}}^n - u_j^{n+\frac{1}{2}}
    }{\Delta x_j^{\pm}}
\end{align*}
be the spatial differences in the two intervals $[x_j, x_{j+\frac{1}{2}}]$ and
$[x_{j-\frac{1}{2}}, x_j]$, respectively.  The weighted average of the spatial
difference can then be calculated with
\begin{align*}
  (u_x^w)_j^{n+\frac{1}{2}} =
  W\left(
    (u_{x-})_j^{n+\frac{1}{2}}, (u_{x+})_j^{n+\frac{1}{2}}
  \right)
\end{align*}
where $W$ is the weighting function.  When there is discontinuity in $u$
between $x_{j-\frac{1}{2}}$ and $x_{j+\frac{1}{2}}$, the weighting function
should detect it and approximate the spatial differencing properly.  A choice
of the weighting function is
\begin{align}
  W_{\alpha} =
    \frac{|u_{x+}|^{\alpha}u_{x-} + |u_{x-}|^{\alpha}u_{x+}}
         {|u_{x-}|^{\alpha} + |u_{x+}|^{\alpha}},
  \label{e:formula:wa}
\end{align}
where $\alpha \in \mathbb{R}$ is a constant parameter and $|u_{x-}|^{\alpha} +
|u_{x+}|^{\alpha} > 0$.  $\alpha$ is often picked as a positive integer, e.g.,
2, for saving computation cycles.  For non-linear equations or discontinuous
initial conditions, a weighting function, e.g., Eq.~(\ref{e:formula:wa}), is
necessary to keep the solution from diverging.

\section{CFL Insensitive Scheme}
\label{s:cese:ctau}

\section{Local Time-Stepping}
\label{s:cese:lts}

\section{Boundary-Condition Treatment}
\label{s:cese:bc}

%%%%%%%%%%%%%%%%%%%%%%%%%%%%%%%%%%%%%%%%%%%%%%%%%%%%%%%%%%%%%%%%%%%%%%%%%%
%%
\chapter{Solving Systems of Equations}
\label{c:syseq}
%%
%%%%%%%%%%%%%%%%%%%%%%%%%%%%%%%%%%%%%%%%%%%%%%%%%%%%%%%%%%%%%%%%%%%%%%%%%%

\section{The One-Dimensional Euler Equations}

%%%%%%%%%%%%%%%%%%%%%%%%%%%%%%%%%%%%%%%%%%%%%%%%%%%%%%%%%%%%%%%%%%%%%%%%%%
%%
\chapter{Two- and Three-Dimensional Unstructured Mesh}
\label{c:ustm}
%%
%%%%%%%%%%%%%%%%%%%%%%%%%%%%%%%%%%%%%%%%%%%%%%%%%%%%%%%%%%%%%%%%%%%%%%%%%%

%%%%%%%%%%%%%%%%%%%%%%%%%%%%%%%%%%%%%%%%%%%%%%%%%%%%%%%%%%%%%%%%%%%%%%%%%%
%%
\clearpage
\addcontentsline{toc}{chapter}{Bibliography}
%%
%%%%%%%%%%%%%%%%%%%%%%%%%%%%%%%%%%%%%%%%%%%%%%%%%%%%%%%%%%%%%%%%%%%%%%%%%%

%\bibliographystyle{myunsrtnat} % no sort (order in appearance)
\bibliographystyle{myplainnat} % sort by author
\bibliography{turgon_main}

\end{document}

% vim: set ai et sw=2 ts=2 tw=79:
