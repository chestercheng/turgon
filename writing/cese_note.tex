\documentclass{turgon}

%\usepackage{lmodern}

\usepackage[printwatermark]{xwatermark}
\newwatermark[allpages,color=black!15,angle=55,scale=5,xpos=0,ypos=0]%
{DRAFT}

%\doublespacing
\linespread{1.2}

\title{
%
Code Development for the Space-Time Conservation Element and Solution Element
Method
%
}

\author{
%
Yung-Yu Chen
%
}

%\date{2008.6.4}

\begin{document}

\maketitle

\tableofcontents

%%%%%%%%%%%%%%%%%%%%%%%%%%%%%%%%%%%%%%%%%%%%%%%%%%%%%%%%%%%%%%%%%%%%%%%%%%
%%
\chapter*{Introduction}
\addcontentsline{toc}{chapter}{Introduction}
%%
%%%%%%%%%%%%%%%%%%%%%%%%%%%%%%%%%%%%%%%%%%%%%%%%%%%%%%%%%%%%%%%%%%%%%%%%%%

Numerical computation, facilitated by the advancement of digital computers,
enables all kinds of simulation, and solving non-linear hyperbolic partial
differential equations (PDEs) is an important one.  Hyperbolic PDEs come from
conservation laws \citep{lax_hyperbolic_1973}.  Except numerical simulation,
there is not a cost-effective way to analyze real-world problems governed by
non-linear hyperbolic PDEs.

Physics, mathematics, and computer code are the three keys to the solution.
The physical insight guides us what we want from the problems.  The
mathematical expression gives us tools.  The computer code materializes the
numerical solution.  In between mathematics and computer program, there is a
staging area called numerical method.  It uses mathematics to describe how the
computer program should solve the problems, but is formulated to enable the
most efficient programming.  It must be emphasized that the numerical
simulation demands utmost runtime performance.  An efficient numerical
formulation is a decisive factor for the usefulness of a code.

This note takes a practical approach for numerical solution of hyperbolic PDEs.
The performance demand significantly restricts how the computer program may be
constructed, and then calls for specially-tuned practices of software
engineering.  The computer program is all the tangibles of what produces the
numerical solution.  No matter how beautiful the formulation is made, a messy
code won't deliver quality results.  This is the first side of the
practicality.

The other side is the focus on the space-time conservation element and solution
element (CESE) method.  The strong non-linearity in conservation laws is
notoriously challenging, but the CESE method treats it well by incorporating
the space-time conservation.  It also takes into account the complex geometry
in the two- and three-dimensional space from the beginning.  Simplices are the
most basic set-up in multi-dimensional space, while extension to other mesh
elements has been done in a comprehensive way, e.g., the use of unstructured
spatial meshes of mixed elements.

In the end, this note will help the readers to understand the CESE method, the
mathematics behind it, and the coding and engineering skills required to
materialize it.

%%%%%%%%%%%%%%%%%%%%%%%%%%%%%%%%%%%%%%%%%%%%%%%%%%%%%%%%%%%%%%%%%%%%%%%%%%
%%
\chapter{Hyperbolic Partial Differential Equations}
\label{c:hyper}
%%
%%%%%%%%%%%%%%%%%%%%%%%%%%%%%%%%%%%%%%%%%%%%%%%%%%%%%%%%%%%%%%%%%%%%%%%%%%

To be determined.

% Possible contents:
% * d'Alembert solution
% * Method of characteristics
% * Non-linear equations
%   * Burger's equations
% * Riemann-invariant
%   * The shallow-water equations
%   * The Euler equations
% * Weak solutions

%%%%%%%%%%%%%%%%%%%%%%%%%%%%%%%%%%%%%%%%%%%%%%%%%%%%%%%%%%%%%%%%%%%%%%%%%%
%%
\chapter{The Conservation Element and Solution Element Method}
\label{c:cese}
%%
%%%%%%%%%%%%%%%%%%%%%%%%%%%%%%%%%%%%%%%%%%%%%%%%%%%%%%%%%%%%%%%%%%%%%%%%%%

% TODO: reorder to a-\mu, a-\varepsion, c, c-\tau, and non-uniform grid.

The conservation element and solution element (CESE) method solves conservation
laws, which can be written in the following form in one-dimensional space
\begin{align}
  \frac{\partial u}{\partial t} + \frac{\partial f(u)}{\partial x} = 0
  .
  \label{e:cese:1d_pde}
\end{align}
$u$ is the dependent solution variable, $f(u)$ a function, and $(x, t)$ the
independent variables.  The space-time $(x, t)$ denotes the two axes of
two-dimensional Euclidean space.  By letting $\mathbf{h} \defeq (f(u),u)$ in
the space-time, write Eq.~(\ref{e:cese:1d_pde}) to $\nabla\cdot\mathbf{h} = 0$.
By assuming the divergence-free condition applies everywhere in the control
volume $V$, write
\begin{align*}
  \int_V\nabla\cdot\mathbf{h}\dif v = 0
  .
\end{align*}
By applying the divergence theorem, the differential equation is cast into an
integral equation over the control surface $S(V)$ surrounding $V$
\begin{align}
  \oint_{S(V)}\mathbf{h}\cdot\dif\hat{\sigma} = 0
  .  \label{e:cese:1d_integral_form}
\end{align}
$\dif\hat{\sigma}$ is the infinitesimal surface vector.
Equation (\ref{e:cese:1d_integral_form}) is what the CESE method solves
\citep{chang_method_1995}.

The space-time in critically important to the CESE method.  The method is
obtained by enforcing flux conservation in space-time.  The value of both
solution variables and their derivatives are solved.  It uses a compact stencil
that defines two entities: the conservation elements (CEs) and solution
elements (SEs).  Space-time invariants are used to minimize numerical
dissipation, but the characteristics-based methods are not used for obtaining
solution.  Ad hoc treatments are avoided as much as possible.

\begin{figure}[htbp]
\centering
  \includegraphics{cce.eps}
  \caption{A compounded conservation element (CCE), the area enclosed by the
  {\color{red} red} dots of {\color{red} $\square\mathrm{BCEF}$}, contains two
  basic conservation elements (BCEs), the area enclosed by the {\color{blue}
  blue} dots of {\color{blue} $\square\mathrm{ABCD}$} and {\color{blue}
  $\square\mathrm{ADEF}$}.}
  \label{f:cce}
\end{figure}

The CEs discretize the space-time for the integral equation to be solved
(Eq.~\ref{e:cese:1d_integral_form})).  $\mathrm{CE}(j,n)$ denotes a single CE
associated with the grid point $(x_j, t^n)$.  In the CE, the conservation of
$\mathbf{h}$ is approximated as
\begin{align}
  \oint_{S(\mathrm{CE})}\mathbf{h}^*\cdot d\hat{\sigma} = 0
  \label{e:conserv_of_approx_h}
\end{align}
where $\mathbf{h}^*$ denotes the approximation of $\mathbf{h}$.  A CE defined
like that is a compounded conservation element (CCE), and consists of two
adjacent basic conservation elements (BCEs) $\mathrm{CE}_-$ and
$\mathrm{CE}_+$, as shown in \figurename~\ref{f:cce}.  Equation
(\ref{e:conserv_of_approx_h}) holds in both CCEs and BCEs, i.e.,
\begin{align*}
  \oint_{S(\mathrm{CE}_\pm)}\mathbf{h}^*\cdot d\hat{\sigma} = 0 .
\end{align*}
$S(\mathrm{CE}_{\pm})$ is the boundary surface surrounding $\mathrm{CE}_{\pm}$.

\begin{figure}[hbtp]
  \centering
  \includegraphics{cese_marching.eps}
  \caption{Time-marching the solution by using the cross-shaped solution
  elements.  The {\color{red} red} dotted crosses mark the SEs at $t=t^0$.  The
  {\color{blue} blue} dotted crosses mark the SEs at $t=t^{1/2}$.  The
  {\color{orange} orange} dotted crosses mark the SEs at $t=t^1$.  The bigger
  dots at the horizontal middle points of the crosses are the solution points.
  The arrows show how the solution variables, that are defined at the solution
  points, at the previous half time step propagate to those at the next half
  time step.  The solution points at the boundary, $x=x_0$ and $x=x_4$, need to
  be updated by boundary-condition treatments, rather than the CESE method
  scheme.}
  \label{f:cese_marching}
\end{figure}

The SEs determine $\mathbf{h}^*$.  There is more than one way to define SEs,
while an effective and consistent approach is shown in
\figurename~\ref{f:cese_marching}.  Let $\mathrm{SE}(j,n)$ denote the SE
associated with the grid point $(j,n)$, which is the cross-shaped mark enclosed
by the dotted line.  The solution variable approximation is written as
\begin{align*}
  u^*(x,t;j,n) = u_j^n + (u_x)_j^n(x-x_j) + (u_t)_j^n(t-t^n) .
\end{align*}
The grid point $(x_j, t^n)$ is used as the solution point.  $u_j^n$,
$(u_x)_j^n$, and $(u_t)_j^n$ hold constant in $\mathrm{SE}(j,n)$.  It should be
noted that every CE is surrounded by SEs.  Fluxes evaluated through the CE
boundary depends only on the approximation within SEs.  To proceed, write
\begin{align*}
  \frac{\partial u^*(x,t;j,n)}{\partial x} = (u_x)_j^n, \quad
  \frac{\partial u^*(x,t;j,n)}{\partial t} = (u_t)_j^n .
\end{align*}
Substitute the approximated solution variable $u^*$ back to the original
differential equation (Eq.~(\ref{e:cese:1d_pde})) and obtain the relation
between $(u_x)_j^n$ and $(u_t)_j^n$ as
\begin{align*}
                  (u_t)_j^n + (f_u)_j^n(u_x)_j^n = 0
  \;\Rightarrow\; (u_t)_j^n = -(f_u)_j^n(u_x)_j^n .
\end{align*}
The approximated solution variable $u^*$ is then rewritten as
\begin{align*}
  u^*(x,t;j,n) = u_j^n + (u_x)_j^n\left[(x-x_j) - (f_u)_j^n(t-t^n)\right] .
\end{align*}
Similarly, the approximated function
\begin{align*}
  f^*(x,t;j,n) = f_j^n + (f_x)_j^n(x-x_j) + (f_t)_j^n(t-t^n)
\end{align*}
is rewritten as
\begin{align*}
  f^*(x,t;j,n) = f_j^n + (f_u)_j^n (u_x)_j^n \left[
    (x-x_j) - (f_u)_j^n(t-t^n)
  \right] .
\end{align*}

\begin{figure}[hbtp]
  \centering
  \includegraphics{se_flux.eps}
  \caption{Space-time flux at the boundary of $\mathrm{CE}(j, n+\frac{1}{2})$
  defined by $\mathrm{SE}(j-\frac{1}{2}, n)$ ({\color{red}red}),
  $\mathrm{SE}(j+\frac{1}{2}, n)$ ({\color{blue} blue}), and $\mathrm{SE}(j,
  n+\frac{1}{2})$ ({\color{orange} orange}).  $x_j$ denotes the grid point of
  the $j$-th SE, $x_j^{\pm}$ the right and left end points, $x_j^s$ the
  solution point.  $\Delta x_j^+ \defeq x_j^+ - x_j$ and $\Delta x_j^- \defeq
  x_j - x_j^-$ are the length of the right and left arms of the $j$-th SE.
  $(\mathbf{h^*})_{j,\pm}^n$ and $(\mathbf{h^*})_{j}^{n,+}$ are the arithmetic
  average of the right, left, and upper arm of $\mathrm{SE}(j,n)$,
  respectively.}
  \label{f:se_flux}
\end{figure}

\begin{figure}[hbtp]
  \centering
  \includegraphics{nonuni_se.eps}
  \caption{The SE definition for a non-uniform one-dimensional grid.  The
  cross-shaped marks are the SEs, and the solid dots are the associated
  solution points.}
  \label{f:nonuni_se}
\end{figure}

The formula for the solution variable $u$ will be obtained by enforcing the
space-time flux conservation over the CCE as shown in Fig.~\ref{f:se_flux}.
Note that for consistent first-order approximation, the solution point must be
at the center of the SE.  In a non-uniform grid, the solution points may not
collocate with grid points (see Fig.~\ref{f:nonuni_se}).  The approximation
formulae should be changed to
\begin{align*}
  u^*(x,t;j,n) &= u_j^n + (u_x)_j^n \left[
    (x-x_j^s) - (f_u)_j^n(t-t^n)
  \right] , \\
  f^*(x,t;j,n) &= f_j^n + (f_u)_j^n (u_x)_j^n \left[
    (x-x_j^s) - (f_u)_j^n(t-t^n)
  \right] .
\end{align*}
The formula for the solution variable $u$ is found to be
\begin{align}
  u_j^{n+\frac{1}{2}}
    = \frac{1}{\Delta x_j}\left\{
      (u^*)_{j-\frac{1}{2},+}^n \Delta x_{j-\frac{1}{2}}^+
    + (u^*)_{j+\frac{1}{2},-}^n \Delta x_{j+\frac{1}{2}}^-
    + \frac{\Delta t}{2} \left[
        (f^*)_{j-\frac{1}{2}}^{n,+} 
      - (f^*)_{j+\frac{1}{2}}^{n,+}
      \right]
      \right\}
  \label{e:formula:u}
\end{align}
where
\begin{align*}
  (u^*)_{j\mp\frac{1}{2},\pm}^n
   &= u_{j\mp\frac{1}{2}}^n
    + (u_x)_{j\mp\frac{1}{2}}^n
      \left( x_{j\mp\frac{1}{2}}
           \pm \frac{1}{2} \Delta x_{j\mp\frac{1}{2}}^{\pm}
           - x_{j\mp\frac{1}{2}}^s \right), \\
  (f^*)_{j\mp\frac{1}{2}}^{n,\pm}
   &= f_{j\mp\frac{1}{2}}^n
    + (f_u)_{j\mp\frac{1}{2}}^n(u_x)_{j\mp\frac{1}{2}}^n
      \left[x_{j\mp\frac{1}{2}} - x_{j\mp\frac{1}{2}}^s
          - (f_u)_{j\mp\frac{1}{2}}^n \frac{\Delta t}{4}
      \right] .
\end{align*}
See Fig.~\ref{f:nonuni_se} for the definition of the left-hand side of the
above equations.

The first-order derivative $u_x$ needs another formula.  Multiple schemes are
available to build the formula.  A convenient way to build it is
\begin{align}
  (u_x)_j^{n+\frac{1}{2}} = \frac{
    (u')_{j+\frac{1}{2}}^n - (u')_{j-\frac{1}{2}}^n
  }{\Delta x_j}
  \label{e:formula:ux:c}
\end{align}
where
\begin{align*}
  (u')_{j\pm\frac{1}{2}}^n &\defeq
      u_{j\pm\frac{1}{2}}^n
    + (u_x)_{j\pm\frac{1}{2}}^n \left[
        x_{j\pm\frac{1}{2}} - x_{j\pm\frac{1}{2}}^s
      - (f_u)_{j\pm\frac{1}{2}}^n \frac{\Delta t}{2} \right]
\end{align*}
are the Taylor expansion to $t^{n+\frac{1}{2}}$ with respect to
$\mathrm{SE}(j\pm\frac{1}{2}, n)$.

Equations (\ref{e:formula:u}) and (\ref{e:formula:ux:c}) form the $c$-scheme,
which is our starting point for discussing the CESE method.

\section{CFL Insensitive Schemes}
\label{s:ctau}

\section{Local Time-Stepping}
\label{s:lts}

To be determined.

%%%%%%%%%%%%%%%%%%%%%%%%%%%%%%%%%%%%%%%%%%%%%%%%%%%%%%%%%%%%%%%%%%%%%%%%%%
%%
\chapter{Solving Systems of Equations}
\label{c:system_eqn}
%%
%%%%%%%%%%%%%%%%%%%%%%%%%%%%%%%%%%%%%%%%%%%%%%%%%%%%%%%%%%%%%%%%%%%%%%%%%%

\section{The One-Dimensional Euler Equations}

%%%%%%%%%%%%%%%%%%%%%%%%%%%%%%%%%%%%%%%%%%%%%%%%%%%%%%%%%%%%%%%%%%%%%%%%%%
%%
\chapter{Two- and Three-Dimensional Unstructured Mesh}
\label{c:ustmesh}
%%
%%%%%%%%%%%%%%%%%%%%%%%%%%%%%%%%%%%%%%%%%%%%%%%%%%%%%%%%%%%%%%%%%%%%%%%%%%

%%%%%%%%%%%%%%%%%%%%%%%%%%%%%%%%%%%%%%%%%%%%%%%%%%%%%%%%%%%%%%%%%%%%%%%%%%
%%
\clearpage
\addcontentsline{toc}{chapter}{Bibliography}
%%
%%%%%%%%%%%%%%%%%%%%%%%%%%%%%%%%%%%%%%%%%%%%%%%%%%%%%%%%%%%%%%%%%%%%%%%%%%

%\bibliographystyle{myunsrtnat} % no sort (order in appearance)
\bibliographystyle{myplainnat} % sort by author
\bibliography{turgon_main}

\end{document}

% vim: set ai et sw=2 ts=2 tw=79:
